\documentclass[11pt,a4paper,finnish,oneside]{article}
\usepackage[utf8]{inputenc}     % Linux
%\usepackage[ansinew]{inputenc} % Windows 
\usepackage[T1]{fontenc}
\usepackage[finnish]{babel}
\usepackage[left=4cm,right=4cm,top=3cm,bottom=3cm]{geometry}
\usepackage{graphicx}
\usepackage{color}
\usepackage{epstopdf}
\usepackage[table]{xcolor}
\usepackage{hyperref}
\hypersetup{colorlinks=true, linkcolor=black, urlcolor=blue}


\usepackage{subfig}



\sloppy
\definecolor{lightgray}{gray}{0.5}
\setlength{\parindent}{0pt}

\begin{document}
\begin{titlepage}  
\title{Graduaiheiden hallintajärjestelmä}
\author{Mikko Kahri\\ (tsoha)}
\date{\today}
\maketitle    
\tableofcontents
\end{titlepage}    


\section{Johdanto}

\begin{par}
Hyvä graduaihe on monen toimijan keskinäisen koordinaation tuote. Graduaiheiden hallintajärjestelmä on luotu helpottamaan tätä koordinaatiota. Järjestelmä mahdollistaa opiskelijoiden ja ohjaajien selvittää, millaisia graduja on tehty, tai työn alla, ja kenen ohjauksessa. Nämä tiedot taas auttavat aiheiden ja ohjaajien etsinnässä, ja mm. päällekkäisyyksien välttämisessä.
\end{par}\vspace{1em}

\begin{par}
Ohjaaja lisää opiskelijan kanssa gradunteon aloittamisesta sovittuaan gradun aiheen järjestelmään, ja voi sen jälkeen muokata sitä halunsa mukaan. Ohjaaja myös luokitelee aiheen johonkin/joihinkin yleistasoiseen luokkaan, ja liittää siihen tietoja sen edistymisestä. Näistä tiedoista osa on ainoastaan ohjaajien käytettävissä.
\end{par}\vspace{1em}

\begin{par}
Järjestelmä toteutetaan tietojenkäsittelytieteen laitoksen users-palvelimella Tomcat- tai Apache-palvelimen alla. Alustajärjestelmän on tuettava PHP -ohjelmointikieltä ja järjestelmän käyttäjän selaimen javascriptiä. Käytettävän tietokannan on oltava PostgreSQL olio-relaatiotietokannan.
\end{par}

\section{Järjestelmän käyttäjät}
\subsection{Käyttötapauskaavio}


\includegraphics [width=6in]{useCase1.png}
\vspace{2em}

\subsection{Käyttäjäryhmät}
\begin{par}
\begin{description}
\item[Opiskelijalla]tarkoitetaan käytännössä ketä tahansa henkilöä, jota kiinnostaa tietää, millaisia graduja on tehty ja/tai tekeillä. Nämä henkilöt ovat potentiaalisia tai aktuaalisia opiskelijoita (tai heidän edustajiaan).
\item[Ohjaaja] on laitoksen graduohjaaksi hyväksymä henkilö, jolla on ollut/ on tällä hetkellä graduja ohjattavanaan, ja jonka ohjaajan ominaisuudessaan tarvitsee merkitä järjestelmään graduaiheita sekä päivittää niitten tietoja.
\end{description}
\end{par}
\subsection*{Käyttötapauskuvaukset}

\begin{par}

\subsubsection{Opiskelijan käyttötapaukset}

\begin{description}
  \item[Aiheiden hakeminen]\hfill\\Opiskelija hakee graduaiheita erityisalan tai ohjaajan perusteella. Hakutuloksena otsikoita sekä vuosilukuja.
  \item[Aiheen katselu]\hfill\\Opiskelija katsoo graduaiheeseen liittyvät tiedot, jotka hänen on mahdollista nähdä. 
\end{description}

\subsubsection{Ohjaajan käyttötapaukset}

\begin{description}
  \item[Aiheen lisäys]\hfill\\Ohjaaja lisää (yhteistyössä opiskelijan kanssa) graduaiheen, jonka on ottanut ohjattavakseen, liittäen siihen tiedon sen tekijästä sekä siihen liittyvistä eriyisaloista. Otsikko ja suunniteltua sisältöä valaiseva kuvaus.

  \item[Aiheen muokkaus/poisto]\hfill\\Ohjaaja muokkaa aiheen otsikkoa, kuvausta tai muita sen tietoja. Ohjaaja voi myös poistaa aiheen.

  \item[Aiheen historiatietojen katseleminen]\hfill\\Ohjaaja katsoo aiheeseen liittyviä historiatietoja. Ajankohtia kuten graduseminaarin suorittamisaika, gradun tarkastukseenjättämisaika jne. Ohjaajien tekemiä muistiinpanoja.
  
    \item[Aiheen historiatietojen muokkaus/lisäys/poisto]\hfill\\Ohjaaja muokkaa, lisää tai poistaa historiatietoja. 
  
\end{description}
Muita tapauksia: Kirjautuminen, rekisteröityminen, aiheiden haku, aiheen katselu.

\end{par}\vspace{1em}
\section{Järjestelmän tietosisältö}
\subsection{Käsitemalli}
\begin{center}
\includegraphics [width=4in]{infoContent1.png}
\end{center}
\subsection{Tietokohteiden kuvaukset}
    \begin{tabular}{ | p{3cm} | p{3cm} | p{6cm} |}
    \multicolumn{3}{l}{\textbf{Edistymistapahtuma}} \\ \hline
    {\small Attribuutti} & {\small Arvojoukko} & {\small Kuvailu}\\ \hline
    Aika & Päiväys & Tapahtuman ajankohta\\ \hline
    Merkitsijä & Ohjaajat & Tapahtuman merkinnyt ohjaaja \\ \hline
    Kommentti & Merkkijono, rajaton pituus & Kommentti, jossa lisätietoa tapahtumasta \\\hline
    \multicolumn{3}{l}{} \\

    \end{tabular}
Graduaiheeseen liittyvä historiatapahtuma, kertoo gradun edistymisestä. Graduaiheeseen voi liittyä useita tapahtumia, mutta tapahtumaan vain yksi graduaihe. Kuhunkin tapahtumaan liittyy tapahtumatyyppi sekä ohjaaja. Ohjaaja voi liittää tapahtumaan kommentin.

\vspace{2em}
    \begin{tabular}{ | p{3cm} | p{3cm} | p{6cm} |}
    \multicolumn{3}{l}{\textbf{Tapahtumatyyppi}} \\ \hline
    {\small Attribuutti} & {\small Arvojoukko} & {\small Kuvailu}\\ \hline
    Nimi & Merkkijono, max. 90 merkkiä & Edistymistapahtuman tyyppi\\ \hline
    \multicolumn{3}{l}{} \\

    \end{tabular}
Merkityksellinen tapahtuma gradun edistymisessä, esim. graduseminaarin päättyminen, gradun tarkastettavaksi jättö ja valmistuminen.

\vspace{2em}
    \begin{tabular}{ | p{3cm} | p{3cm} | p{6cm} |}
    \multicolumn{3}{l}{\textbf{Aihe}} \\ \hline
    {\small Attribuutti} & {\small Arvojoukko} & {\small Kuvailu}\\ \hline
    Otsikko & Merkkijono, max. 300 merkkiä & Graduaihetta kuvaava otsikko\\ \hline
    Kuvaus & Merkkijono, max 3000 merkkiä & Otsikkoa täsmentävä kuvaus \\ \hline
    Opiskelijanumero & Kokonaisluku, 9 numeromerkkiä & Tekijän opiskelijanumero \\\hline
    Nimi & Merkkijono, max. 80 merkkiä & Tekijän nimi \\\hline
    Luotu & Päiväys & Aiheen luontiajankohta\\\hline
    Luoja & Ohjaajat & Aiheen luonut ohjaaja\\\hline
    Linkki & verkko-osoite & Linkki valmiin gradun tietoihin\\\hline
    
    \multicolumn{3}{l}{} \\

    \end{tabular}
Aihe on keskeinen tietokohde graduaiheiden hallintajärjestelmässä. Aiheeseen liittyy historiatapahtumia sekä yksi tai useampia erityisaloja. Jokainen aihe on jonkun ohjaajan luoma. Gradulla voi olla useita ohjaajia.

\vspace{2em}
    \begin{tabular}{ | p{3cm} | p{3cm} | p{6cm} |}
    \multicolumn{3}{l}{\textbf{Ohjaaja}} \\ \hline
    {\small Attribuutti} & {\small Arvojoukko} & {\small Kuvailu}\\ \hline
    Etunimi & max. 40 merkkiä & Ohjaajan etunimi\\ \hline
    Sukunimi & max. 40 merkkiä & Ohjaajan sukunimi\\ \hline
    Salasana & Merkkijono max. 100 merkkiä & Salasana kirjautumista varten \\ \hline    
    Sähköposti & Merkkijono max. 80 merkkiä & Ohjaajan s-postiosoite \\ \hline
    \multicolumn{3}{l}{} \\

    \end{tabular}
Laitoksen graduohjaajaksi hyväksymä henkilö. Ohjaajalla voi olla monta gradua ohjattavanaan.

\vspace{2em}
    \begin{tabular}{ | p{3cm} | p{3cm} | p{6cm} |}
    \multicolumn{3}{l}{\textbf{Tutkimusala}} \\ \hline
    {\small Attribuutti} & {\small Arvojoukko} & {\small Kuvailu}\\ \hline
    Nimi & max. 90 merkkiä & Tutkimusalan nimi\\ \hline
    \multicolumn{3}{l}{} \\

    \end{tabular}
Yksi gradu voi liittyä useaan tutkimusalaan, ja tutkimusala useaan graduun.

\vspace{20em}
\section{Relaatiotietokantakaavio}

\includegraphics [width=7in]{classDiagram1.png}

\section{Järjestelmän yleisrakenne}

\begin{par}
Tietokantasovellusta tehdessä on noudatettu MVC-mallia. Kontrollerit, näkymät ja mallit sijaitsevat hakemistoissa \verb|controllers, views| ja \verb|models|. Käytetyt apukirjastot on sijoitettu hakemistoon \verb|lib| ja asetukset löytyvät kansion \verb|config| tiedostoista. (Ohjaajaksi rekisteröitymisessä vaadittava rekisteröintitunnus sekä Valitronin suomenkielisten palautetekstien polku on asetettu tiedostossa \verb|lib/base_controller.php.)| Tietokannan taulujen luontiin käytetyt komennot löytyvät kansiosta \verb|sql|. Järjestelmän hakutoiminnon toteutuksessa käytetty javastcript -koodi löytyy tiedostosta \verb|assets/js/site.js|.
\end{par}\vspace{1em}
\begin{par}
Tiedostojen, luokkien ja funktioiden nimissä on käytetty englannin kieltä, muuttujien nimissä ja reiteissä suomen kieltä. Malliluokkien nimet ovat englanninkielisiä käännöksiä tietokannan suomenkielisten taulujen nimistä. Kontrolleriluokkien nimet ovat alkuosaltaan viittauksia näihin nimiin, loppuosan ollessa aina \verb|controller|. Tallentamiseen liittyvissä funktioitten nimissä on mahdollisuuksien mukaan käytetty sanaa \verb|save|, taulujen muuttamiseen liittyvien funktioiden nimissä sanaa \verb|update|, ja poistamiseen liittyvissä sanaa \verb|destroy|.  
\end{par}\vspace{1em}

\section{Käyttöliittymä ja järjestelmän komponentit}

\includegraphics [width=5in]{komponentit1.png}
\vspace{2em}
\begin{par}
Muokkaussivulla 1 lisätään ja poistetaan aiheeseen ohjaajia, tutkimusaloja ja tapahtumia. Poiston yhteydessä tapahtuu uudelleenohjaus samalle sivulle. Muokkaussivulla 2 muokataan aiheen varsinaisia tietoja -- sen nimeä, tekijää ja kuvausta. Tapahtuman muokkaussivulla ohjaaja voi tarkastella ja tehdä tapahtumaan liittyviä muistiinpanoja.\end{par}
 \begin{par}\vspace{1em}
Sivustolla on navigaatiopalkki, jonka kautta jokainen käyttäjä pääsee mistä tahansa aloitussivulle, aiheitten selaissivulle, kirjautumissivulle tai rekisteröitymissivulle. Aiheen poisto on mahdollista muokkaussivuilla 1 ja 2, ja aiheen poiston jälkeen tapahtuu uudelleenohjaus aiheitten selaussivulle.
\end{par}

\section{Asennustiedot}
\begin{par}
Asenna sovellus kopioimalla sen tiedostot palvelimen nettiin näkyvään hakemistoon (esim. usersin htdocs-hakemisto). Sinun täytyy myös asentaa \href{https://getcomposer.org/}{Composer} (voit halutessasi katsoa mallia tiedostosta \verb|bootstrap.sh|) sekä \href{https://github.com/vlucas/valitron}{Valitron}. Aseta myös tietokannan yhteystiedot oikeiksi tiedostossa \verb|config/database.php|.  Lisätietoja, ks. \href{http://tsoha.github.io/}{http://tsoha.github.io/}.
\end{par}




\section{Käynnistys- / käyttöohje}
\begin{par}
Harjoitustyö on asennettuna osoitteessa \href{http://mkahri.users.cs.helsinki.fi/tsoha/}{mkahri.users.cs.helsinki.fi/tsoha/}. Järjestelmään voi kirjautua koekäyttöä varten esim. seuraavilla (kuvitteellisilla) käyttäjätiedoilla:

\vspace{2em}

    \begin{tabular}
    { | p{8cm} | p{4cm} |}
    \hline
    sähköposti & salasana\\ \hline
    olli.ohjausmestari@helsinki.fi & Olli123 \\ \hline
    kalle.z.kakkonen@helsinki.fi & Kalle123 \\ \hline
    \end{tabular}

\vspace{2em}

Jos haluaa lisätä järjestelmään uuden ohjaajan, tarvitsee rekisteröitymistunnuksen, joka on tätä kirjoitettaessa \textbf{virallinenohjaaja}. 

\end{par}

\section{Testaus ja puutteet}
\subsection{Testaus}
\begin{par}
Järjestelmän testaus on tämän harjoitustyön osalta ollut enimmäkseen sitä, että olen aika-ajoin kiertänyt (kirjautuneena ja kirjautumattomana) painamassa kaikkien lomakkeitten kaikki nappulat erilaisilla lomakkeitten sisällöillä (ja ilman sisältöä) ja tarkkaillut, että toiminta on ollut johdonmukaista. En ole tietoinen, että esiintyisi häiriötiloihin joutumista tai muuta yllättävää käyttäytymistä.
\end{par}

\subsection{Bugit ja puutteet}
\begin{par}
Listaussivun hakutoiminnon soisi olevan epäsensitiivisen pienille/suurille alkukirjaimille, mutta näin ei kuitenkaan ole ("relaatiotietokannat"\ antaa eri hakutuloksen kuin "Relaatiotietokannat", esimerkiksi.) Olisi intuitiivista, ettei tapahtumatyyppiä "valmis" \ voisi lisätä kahdesti, mutta sen kuitenkin voi.
\end{par}\vspace{1em}
\begin{par}

Graduaiheeseen liittyvien tapahtumien aikatiedoiksi tulee nykyisellään aina aika, jolloin tapahtuma on kirjattu järjestelmään. Toiminnallisuus, jolla voi kirjata tapahtumia jälkikäteen olisi ollut tarpeellinen. Samaten olisivat olleet tarpeellisia toiminnallisuudet, joilla lisätä uudenlaisia edistymistapahtumia ja uusia tutkimusaloja (nykyisellään niitä voi lisätä ainoastaan suorilla komennoilla tietokantaan). Edellisiin liittyen, myös ylläpitäjän käyttäjärooli olisi ollut hyvä luoda. Ohjaaja ei voi itse muokata omia tietojaan. 
\end{par}

\subsection{Jatkokehitysideat}
\begin{par}
Mietin, olisiko ollut parempi, jos sen sijaan, että gradun ohjaaja lisää aiheeseen siihen liittyviä edistymistapahtumia, gradun tekijä olisi saanut tilaisuuden liittää aiheeseensa joitakin sen tekoon liittyviä tietoja. Luulen, että oikeassa elämässä ohjaajat eivät ehdi/halua lisätä jokaisen ohjauspalaverin jälkeen graduaiheiden hallintajärjestelmään siihen liittyviä tietoja. Ideaalissa tapauksessa moiset tiedot tietenkin saisi työajan hallintajärjestelmistä ja toisaalta opintorekisteristä tms. Periaatteessa tämän tyyppiselle järjestelmälle voisi ihan oikeasti olla joillakin laitoksilla/ joissakin tiedekunnissa tarvetta.
\end{par}

\section{Omat kokemukset}
\begin{par}
Kokemus oli myönteinen. Tietokantakurssin asiat tulivat tietokannan rakentamisen ja käytön myötä konkreettisiksi. Kurssi ei ollut vaikea. Haasteet olivat sopivia. Jotkin asiat oli ehkä tehty jopa turhan helpoiksi. En osaisi tehdä uutta vastaavaa sovellusta ilman Tsoha-bootstrap -paketin tarjoamaa "huolenpitoa". Olisi ollut hauska, jos olisi kaiken muun lisäksi saanut kurssilta vielä moisen taidonkin mukaansa.
\end{par}\vspace{1em}

\end{document}
    
